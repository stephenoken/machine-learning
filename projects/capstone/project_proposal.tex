\hypertarget{machine-learning-engineer-nanodegree}{%
\section{Machine Learning Engineer
Nanodegree}\label{machine-learning-engineer-nanodegree}}

\hypertarget{capstone-proposal}{%
\subsection{Capstone Proposal}\label{capstone-proposal}}

Stephen O'Kennedy\\
25 April, 2018

\hypertarget{proposal-identify-and-classify-toxic-online-comments}{%
\subsection{Proposal: Identify and classify toxic online
comments}\label{proposal-identify-and-classify-toxic-online-comments}}

\hypertarget{domain-background}{%
\subsubsection{Domain Background}\label{domain-background}}

Since it's inception the internet has allowed people from most parts of
the world to freely communicate, debate, and collaborate with each other
over a wide range of topics and projects. Platforms like Github,
Hackernews, Twitter, Wikipedia, etc. form the foundations for which
these interactions can take place. Many of these communities have
standards and rules in place to facilitate conversations, and to prevent
these communities from being hijacked, or destroyed by toxic behaviour.
It is becoming increasingly harder to regulate and enforce these
standards. In fact Facebook are currently hiring more and more
moderators to sift through questionable content
\href{http://fortune.com/2018/03/22/human-moderators-facebook-youtube-twitter/}{1}.

\href{https://conversationai.github.io/}{Conversation AI} are working to
provide tools to help improve online
conversation\href{https://www.kaggle.com/c/jigsaw-toxic-comment-classification-challenge}{2}.
One area that they're focusing on is the study of negative online
behaviours, like toxic
comments\href{https://www.kaggle.com/c/jigsaw-toxic-comment-classification-challenge}{2}.
As their Kaggle page states, the current models in use for detecting
toxic comments still make errors, and they don't allow users to be able
to identify the types of toxicity they're interested in finding. For
example some platforms may be fine with comments that contain
profanities.

\hypertarget{problem-statement}{%
\subsubsection{Problem Statement}\label{problem-statement}}

Given a dataset that contains a large number of Wikipedia comments which
have been labelled by human raters for toxic behaviour. We want to
create a model that predicts the probability of different types of
toxicity for each comment.

\hypertarget{datasets-and-inputs}{%
\subsubsection{Datasets and Inputs}\label{datasets-and-inputs}}

We are provided with a dataset in csv format where we have the following
columns: \texttt{id} \texttt{comment\_text} : {[}String{]}
\texttt{toxic} : int \texttt{severe\_toxic} : int \texttt{obscene} : int
\texttt{threat} : int \texttt{insult} : int \texttt{identity\_hate} :
int

The \texttt{comment\_text} column is comment that we want to feed into
our classifier, and the outputs will be will be a vector containing the
probabilities of the comment being one of the \texttt{toxic},
\texttt{obscene}, etc. The training data set can be found here
\href{https://www.kaggle.com/c/8076/download/train.csv.zip}{3}.

\hypertarget{solution-statement}{%
\subsubsection{Solution Statement}\label{solution-statement}}

The solution will involved the development of deep learning algorithm
that uses Keras with TensorFlow being used as the backend. Our aim is to
use a multi-class CNN to process the content of the comments and out put
a ROC AUC score
\href{http://scikit-learn.org/stable/modules/generated/sklearn.metrics.roc_auc_score.html}{4}.
Finally, predictions will be made on the test data set and will be
evaluated on Kaggle.

\hypertarget{benchmark-model}{%
\subsubsection{Benchmark Model}\label{benchmark-model}}

The benchmark score we'll use to compare our model against will be
\texttt{0.982900}. This score was is in the 50th percentile of the
public
leaderboards\href{https://www.kaggle.com/c/8076/publicleaderboarddata.zip}{5},
and was calculated using ROC AUC metric
\href{http://scikit-learn.org/stable/modules/generated/sklearn.metrics.roc_auc_score.html}{4}.

\hypertarget{evaluation-metrics}{%
\subsubsection{Evaluation Metrics}\label{evaluation-metrics}}

Submissions are evaluated by using the ROC AUC metric. Each comment in
the test data set will need to be labeled with the predictions for each
type of toxicity appearing in each comment, and will need to be
submitted to Kaggle.

ROC is the receiver operating characteristic curve. It is a graphical
plot that displays the discrimination threshold of a binary classier,
which is what we'll need to build. Our threshold (\(T\)), which is used
to classify a datapoint as either positive or negative, is by default
set \(0.5\). We take the true positive rate (\(TPR\)) and false positive
rate (\(FPR\)) for all scores and plot a curve. Calculating the AUC
(area under the curve) will reduce the curve down to a single value,
\(1 \ge A < 0\). Where \(A\) is the AUC. If \(A\) is close to \(1.0\)
we've got a perfect classifier, if However it is \(0.5\) of lower than
our classifier is doing little more than guessing
\href{https://en.wikipedia.org/wiki/Receiver_operating_characteristic\#Area_under_the_curve}{6}.

The formula is:

\[
A = \int_{\infty}^{-\infty} TPR(T)FPR'(T)dT = \int_{-\infty}^{\infty}\int_{\infty}^{-\infty}I(T' > T)f_1(T')f_0(T)dT'dT = P(X_1 >X_0)
\]

\hypertarget{project-design}{%
\subsubsection{Project Design}\label{project-design}}

As described in the problem statement, we will be analysing comments
made on wikipedia. We will first need to perform data analysis and get
familiar with the data. There are several questions that come to mind
immediately that I want to understand about the data. For example, are
the classes balanced? Is the data set complete? Do some of the comments
use non standard characters?

Once we've completed our data analysis we will begin text processing.
this will include removing stop words, stemming, tokenising,
vectorising, etc. We would like to point out since we're trying to
process natural language we're going to try different n-gram ranges such
as (2,4) in order to capture toxic phrases. It also worth noting that
we'll need to investigate how to deal with misspellings. Therefore just
by processing the text content of the contents alone would lead us to
implement use grid search to tune the our text vectorising algorithm.

We will then need to define our CNN or MLP architecture. Our initial
plan is to create a reasonably shallow neural network and establish a
baseline model to work from. We model our initial network on something
like Alexnet. We could then use a pre-trained network such as Google's
inception or VGG19. This raises question then, is our training data big
enough for us to retrain our neural network and fine-tune? Or do we
remove the initial convoluted layers, add our own, and bolt on the rest
of the network and slice the end of the network to match the number of
outputs.
